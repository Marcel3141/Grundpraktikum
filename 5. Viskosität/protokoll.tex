\documentclass[12pt, a4paper, twoside]{scrartcl}
 %---- Allgemeine Layout Einstellungen ------------------------------------------

% Für Kopf und Fußzeilen, siehe auch KOMA-Skript Doku
\usepackage[komastyle]{scrpage2}
\pagestyle{scrheadings}
\setheadsepline{0.5pt}[\color{black}]


%Einstellungen für Figuren- und Tabellenbeschriftungen
\setkomafont{captionlabel}{\sffamily\bfseries}
\setcapindent{0em}


%---- Weitere Pakete -----------------------------------------------------------
% Die Pakete sind alle in der TeX Live Distribution enthalten. Wichtige Adressen
% www.ctan.org, www.dante.de

% Sprachunterstützung
\usepackage[ngerman]{babel}

% Benutzung von Umlauten direkt im Text
% entweder "latin1" oder "utf8"
\usepackage[utf8]{inputenc}

% Pakete mit Mathesymbolen und zur Beseitigung von Schwächen der Mathe-Umgebung
\usepackage{latexsym,exscale,stmaryrd,amssymb,amsmath}

% Weitere Symbole
\usepackage[nointegrals]{wasysym}
\usepackage{eurosym}

% Anderes Literaturverzeichnisformat
%\usepackage[square,sort&compress]{natbib}

% Für Farbe
\usepackage{color}

% Zur Graphikausgabe
%Beipiel: \includegraphics[width=\textwidth]{grafik.png}
\usepackage{graphicx}

% Text umfließt Graphiken und Tabellen
% Beispiel:
% \begin{wrapfigure}[Zeilenanzahl]{"l" oder "r"}{breite}
%   \centering
%   \includegraphics[width=...]{grafik}
%   \caption{Beschriftung} 
%   \label{fig:grafik}
% \end{wrapfigure}
\usepackage{wrapfig}

% Mehrere Abbildungen nebeneinander
% Beispiel:
% \begin{figure}[htb]
%   \centering
%   \subfigure[Beschriftung 1\label{fig:label1}]
%   {\includegraphics[width=0.49\textwidth]{grafik1}}
%   \hfill
%   \subfigure[Beschriftung 2\label{fig:label2}]
%   {\includegraphics[width=0.49\textwidth]{grafik2}}
%   \caption{Beschriftung allgemein}
%   \label{fig:label-gesamt}
% \end{figure}
\usepackage{subfigure}

% Caption neben Abbildung
% Beispiel:
% \sidecaptionvpos{figure}{"c" oder "t" oder "b"}
% \begin{SCfigure}[rel. Breite (normalerweise = 1)][hbt]
%   \centering
%   \includegraphics[width=0.5\textwidth]{grafik.png}
%   \caption{Beschreibung}
%   \label{fig:}
% \end{SCfigure}
\usepackage{sidecap}
\usepackage{float}

% Befehl für "Entspricht"-Zeichen
\newcommand{\corresponds}{\ensuremath{\mathrel{\widehat{=}}}}
\newcommand{\folgt}{\ensuremath{\mathrel{\Rightarrow}}}
\newcommand{\equals}{\ensuremath{\mathrel{\Leftrightarrow}}}
\newcommand{\degree}{\ensuremath{\mathrel{^{\circ}}}}

\newcommand{\nn}{\nonumber}
\newcommand{\tn}[1]{\textnormal{#1}}
\newcommand{\D}{\ensuremath{\mathrel{\rm d}}}

\newcommand{\const}{\tn{const}}

\newcommand{\meter}{\ensuremath{\mathrel{\tn m}}}
\newcommand{\kilogramm}{\ensuremath{\mathrel{\tn{kg}}}}
\newcommand{\second}{\ensuremath{\mathrel{\tn s}}}
\newcommand{\sekunde}{\second}

\newcommand{\volt}{\ensuremath{\mathrel{\tn V}}}
\newcommand{\pascal}{\ensuremath{\mathrel{\tn{Pa}}}}
\newcommand{\coulomb}{\ensuremath{\mathrel{\tn C}}}
\newcommand{\newton}{\ensuremath{\mathrel{\tn N}}}
\newcommand{\liter}{\ensuremath{\mathrel{\tn l}}}
\newcommand{\celsius}{\ensuremath{\mathrel{\tn C}}}
\newcommand{\fahrenheit}{\ensuremath{\mathrel{\tn F}}}
\newcommand{\joule}{\ensuremath{\mathrel{\tn J}}}
\newcommand{\kelvin}{\ensuremath{\mathrel{\tn K}}}
\newcommand{\mol}{\ensuremath{\mathrel{\tn{mol}}}}

\newcommand{\kilo}{\ensuremath{\mathrel{\tn k}}}
\newcommand{\hecto}{\ensuremath{\mathrel{\tn h}}}

\newcommand{\centi}{\ensuremath{\mathrel{ \tn c}}}
\newcommand{\milli}{\ensuremath{\mathrel{ \tn m}}}
\newcommand{\micro}{\ensuremath{\mathrel{ \tn\mu }}}



%\newcommand{}{\ensuremath{\mathrel{  }}}
%\newcommand{}{\ensuremath{\mathrel{  }}}
%\newcommand{}{\ensuremath{\mathrel{  }}}


\newcommand{\person}[1]{\textsc{#1}}

 \begin{document}
 %Titelseite u. Inhaltsverzeichnis
\input{title.tex}
\cleardoublepage
\tableofcontents
\cleardoublepage
\setcounter{page}{1}

\section{Einleitung}
\label{sec:einleitung}
Dieser Versuch ist in zwei Teile unterteilt. Im ersten Experiment wird das Phänomen der Kapillarität untersucht, welches  es Flüssigkeiten ermöglicht durch Oberflächenspannung die Schwerkraft zu überwinden. Dadurch wird zum Beispiel der Wassertransport in Pflanzen ermöglicht. Im zweiten Experiment wird die Viskosität untersucht. Diese ist ein Maß für die Zähflüssigkeit einer Flüssigkeit und beeinflusst damit unter anderem ihre Fließgeschwindigkeit.


\section{Theorie}
\label{sec:theorie}

\subsection{Oberflächenspannung}
Der Effekt der Kapillarität wird von Wechselwirkungen auf molekularer Ebene verursacht. Diese können in zwei Arten unterteilt werden. Zum einen in die Kohäsionskräften, die innerhalb der Flüssigkeit wirken und zum anderen in die Adhäsionskräften, Kräften die zwischen der Flüssigkeit und einem Festkörper (Glas) oder einem Gas (Luft) auftreten.\newline
\newline
Die wichtigsten Kohäsionskräfte sind die Van-der-Waals-Kräfte und die Dipol-Dipol-Kräfte. Die Van-der-Waals-Kräfte sind schwache Kräfte die zwischen Molekülen und Atomen wirken. Sie entstehen durch zufällige Ladungsverschiebung innerhalb eines Moleküls, verursacht durch die Bewegung seiner freien Elektronen. Diese Ladungsverschiebungen bewirken, dass ein Molekül kurzzeitig zu einem Dipol wird und so mit anderen Molekülen wechselwirken kann. 
Dipol-Dipol-Kräfte hingegen werden zwischen Molekülen erzeugt, die ein dauerhaftes Dipolmoment besitzen.\newline
\newline
Hält man einen Festkörper in eine Flüssigkeit so entstehen zwischen den Molekülen der Flüssigkeit und den Molekülen des Festkörpers Adhäsionskräfte. Sind diese Adhäsionskräfte größer als die Kohäsionskräfte innerhalb der Flüssigkeit so kann man beobachten wie sich die Flüssigkeit am Rand des Festkörpers hochzieht. Dieser Effekt heißt Kapillarität und ist besonders gut zu beobachten, wenn man eine Glaskapillare in Wasser hält. Man sieht,dass das Wasser im Inneren der Kapillare einen höheren Pegelstand als die Flüssigkeit außerhalb der Kapillare hat. Durch die stärkeren Adhäsionskräfte leistet das Wasser Arbeit entgegen der Gravitationskraft, was den Begriff der Oberflächenspannung $\sigma$ motiviert. \[\textup dW=\sigma\cdot\textup dA\hspace{1cm}\Leftrightarrow\hspace{1cm}\sigma=\frac{\textup dW}{\textup dA}\] Steigt das Wasser in der Kapillare um die noch zu bestimmende Höhe h an, so verändert sich die potentielle Energie um \[\textup dE_{Pot}=m\cdot g\cdot\textup dh=\rho\cdot\pi\cdot R^2\cdot h\cdot g\cdot\textup dh\] und die Oberflächenenergie um \[\textup dE_{O}=-2\cdot\pi\cdot R\cdot\sigma\cdot\textup dh\]Dabei ist $\rho$ die Dichte des Wassers und $R$ der Radius der Kapillare. Die Energieerhaltung liefert 
\begin{align}
\textup dE_{Pot}+\textup dE_{O}&=0 \nonumber\\
\Leftrightarrow\hspace{0.6cm}\pi\cdot R^2\cdot\rho\cdot h \cdot g\cdot\textup dh&=2\pi\cdot R\cdot\sigma\cdot\textup dh \nonumber\\
\Leftrightarrow\hspace{1.5cm} \frac{1}{2}\cdot R\cdot\rho\cdot h\cdot g&=\sigma \label{eq:oberflaechenspannung}
\end{align}\newline

Analog zur Oberflächenspannung zwischen Flüssigkeiten und Luft kann man die allgemeine Grenzflächenspannung $\sigma_{i,k}$ zwischen zwei beliebigen Stoffen $i$ und $k$ definieren.
\begin{figure} [h]
\centering
\includegraphics[scale=0.8]{grenzflaechen.png}
\caption{Zur Definition von Grenzflächen\protect\footnotemark}
\end{figure}
\footnotetext{http://lp.uni-goettingen.de/get/text/5065}

Wie man der Grafik entnehmen kann, können $\sigma_{2,3}$ und $\sigma_{1,3}$ nur in entgegengesetzte Richtung wirken. Also muss
\begin{align*}
 \sigma_{2,3}-\sigma_{1,3}=-\sigma_{1,2}\cdot\cos\theta
\end{align*}
gelten.\newline

Um die Dichten der untersuchten Flüssigkeiten zu bestimmen wird die Mohrsche Waage verwendet.

\begin{figure} [h]
\centering
\includegraphics[scale=0.6]{mwaage.png}
\caption{Die Mohrsche Waage\protect\footnotemark}
\end{figure}
\footnotetext{https://lp.uni-goettingen.de/get/text/3638}

Es wird ein Körper mit bekanntem Volumen in die zu untersuchende Flüssigkeit getaucht. Durch die vom Körper verdrängte Flüssigkeit entsteht ein statischer Auftrieb, aus dem nach dem archimedischen Prinzip die Dichte der Flüssigkeit ermittelt werden. Um den entstehenden Auftrieb zu messen wird der Lastarm mittels Gewichten ausgeglichen.
Für eine Flüssigkeit der Dichte $\rho$ gilt bei Gleichgewicht
\begin{align}
 D_{Auftrieb} &= D_{Gewichte} \nonumber\\
\equals  V\cdot\rho\cdot g \cdot r &= \sum_{i=1}^{n}m_i\cdot g\cdot r_i
\end{align}
Somit gilt für zwei Flüssigkeiten
\begin{align}
\frac{V\cdot\rho_2\cdot g \cdot r}{V\cdot\rho_1\cdot g \cdot r} &= \frac{\sum_{i=1}^{n_2}m_{2,i}\cdot g\cdot r_{2,i}}{\sum_{i=1}^{n_1}m_{1,i}\cdot g\cdot r_{1,i}} \nonumber\\
\equals \rho_2 &= \rho_1 \cdot \frac{\sum_{i=1}^{n_2}m_{2,i}\cdot r_{2,i}}{\sum_{i=1}^{n_1}m_{1,i}\cdot r_{1,i}} \label{eq:mohrsche_waage}
\end{align}


\subsection{Dynamische Viskosität}
Die Viskosität $\eta=v\cdot\rho$ eines Fluids ist ein Maß für dessen Zähflüssigkeit.\linebreak

Es existieren zwei verschiedene Strömungsarten die bei Fluiden auftreten können. Die laminare Strömung, bei der keine Turbolenzen auftreten, die Fluide also in Schichten strömen und die turbulente Strömung, bei der Verwirbelungen auftreten, so dass die einzelnen Schichten eines Fluids untereinander vermischt werden.\newline
Das Verhältnis von Trägheits- zu Zähigkeitskräften wird durch die dimensionslose  Reynoldzahl Re angegeben. Sie ist ein Maß dafür ob eine Strömung laminar oder turbulent ist. Es gilt
\begin{align*}
  \rm{Re}=\frac{\rho\cdot v\cdot d}{\eta},
\end{align*}
,wobei $\rho$ die Dichte des Fluids, $d$ die Länge des Gegenstandes in dem sich die Strömung befindet und $v$ die durchschnittliche Geschwindigkeit bezeichnet.\linebreak
 
Verursacht wird die Viskosität von der inneren Reibung $F_r=-\eta\cdot A\cdot\frac{\rm dv}{\rm dr}=-\eta\cdot 2\pi l r\cdot\frac{\rm dv}{\rm dr}$ die entsteht, wenn eine Flüssigkeit durch ein Rohr fließt. Aufgrund des Druckunterschieds $\delta p=(p_1-p_2)$ wirkt ihr dir Kraft $F_p=(p_1-p_2)\cdot\pi\cdot r^2$ entgegen. Mit der Randbedingung v(R)=0 erhält man
\begin{align*}
\Rightarrow\hspace{1cm}-\eta\cdot2 l\cdot\frac{\rm dv}{\rm dr}&=(p_1-p_2)\cdot r \\
\Leftrightarrow\hspace{1.0cm}-\int_r^R\frac{\rm dv}{\rm dr}\rm dr&=\int_r^R\frac{(p_1-p_2)}{2l\cdot\eta} \\
\Leftrightarrow\hspace{2.35cm}v(r)&=\frac{(p_1-p_2)}{4\eta\cdot l}(R^2-r^2)
\end{align*}
Damit lässt sich die gesamte Flüssigkeitsmenge, die pro Zeiteinheit t durch einen Hohlzylinder fließt angeben durch $\rm dV=2\pi\cdot r\cdot v(r)\cdot t\cdot dt$. Integrieren liefert
\begin{align*}
\Rightarrow \hspace*{1cm}\int_0^R\rm dV\,\rm dr&=2\pi\cdot\frac{(p_1-p_2)}{4l\cdot\eta}\cdot t\cdot\int_0^Rr(R^2-r^2)\,\rm dr \\
\Leftrightarrow \hspace*{2.4cm}V&=\frac{\pi\cdot(p_1-p_2)}{8l\cdot\eta}\cdot R^4\cdot t\rm,
\end{align*}
woraus man die Hagen-Poiseille Gleichung der laminaren Rohrströmung erhält
\begin{align}
\dot V&=\frac{\pi\cdot(p_1-p_2)}{8l\cdot\eta}\cdot R^4 \label{eq:viskositaet}
\end{align}
Diese Gleichung sagt aus, wie der Volumenstrom vom Druckgradienten $(p_1-p_2)/l$ und von der Viskosität des Fluids abhängt. Dies zu wissen ist zum Beispiel bei der Wahl des Motoröls im Auto entscheidend. Da man den Rohrradius in dem das Öl zufließt nicht verändern kann, war es früher notwendig verschiedene Motoröle mit verschiedenen Viskositäten $\eta$ zu verwenden um auf temperaturbedingte Druckunterschiede zu reagieren.\\

%Umformen der Gleichung nach $\eta$ ergibt die Formel für die Viskosität für ein Volumen $V_{h_2}=A\cdot h_2$ das in der Zeit $t_{h_2}$ abfließt. Die Druckdifferenz ist $(p_1-p_2)=\frac{1}{2}\rho g(h_1-h_2)$.
%\begin{align*}
%\eta=\frac{\pi\cdot\rho g(h_1-h_2)\cdot R^4}{16l}\cdot\frac{t_{h_2}}{V_{h_2}}
%\end{align*}

\section{Durchführung}
\label{sec:durchfuehrung}

Um Messfehler durch Verunreinigungen vorzubeugen sind die Kapillare vor jeder Messung gründlich mit 
Lösungsmittel und Wasser zu säubern und anschließend mit der Wasserstrahlpumpe zu trocknen. \\
Der Radius R jeder Kapillare ist mindestens dreimal mit Hilfe des Mikroskops zu bestimmen.\\

\subsection{Teilversuch 1: Kapillarität}
Die zu untersuchenden Flüssigkeiten, destilliertes Wasser, Methanol und Ethylenglykol werden in ausreichender Menge in Bechergläser gegeben. Jede der Kapillaren wird in die Flüssigkeiten getaucht und dann soweit wieder herausgehoben, dass sich der Nullpunkt der Messskala auf gleichem Niveau mit dem Flüssigkeitsspiegel der Flüssigkeit im Becherglas befindet. Mit Hilfe der Skala ist dann der Höhenunterschied $h_kap$ der Flüssigkeitsspiegel zu bestimmen. Dies wird für jede Kapillare und jede Flüssigkeit mindestens dreimal wiederholt. \\
Die Dichte $rho$ der Flüssigkeiten ist mit Hilfe der Mohrschen Waage zu bestimmen, wobei darauf geachtet werden sollte, dass der Probekörper trocken und sauber ist, bevor er in die Flüssigkeit getaucht wird und dass er in diese für die Messung dann vollständig eingetaucht ist.\\

\subsection{Teilversuch 2: Viskosität}
\begin{figure}
\centering
\includegraphics[scale=0.7]{Hagen-Poiseuille.png}
\caption{Versuchsaufbau: $\left( \rm{Hagen-Poiseuillesches Gesetz} \right)$}
\end{figure}
Zu messen sind die Temperatur $T_W$ des destillierten Wassers, die Länge der Kapillaren $l$, sowie das Volumen des Glasgefäßes zwischen den Strichmarken 50 und 45.\\
\begin{itemize}
\item Für jedes der drei Kapillare ist die Ausflusszeit $t_A$ von destilliertem Wasser zwischen den Strichmarken  50 und 45 zu bestimmen.
\item Für die Kapillare mit dem geringsten Durchmesser sind mindestens 10 Werte der Abflusszeit $t \left( h \right)$ $\left( \textrm{in Abhängigkeit der Höhe} h \textrm{der Wassersäule} \right)$ zu bestimmen.
\end{itemize}

\section{Auswertung}
\label{sec:auswertung}

In der Tabelle \ref{tab:l_kap} und Tabelle \ref{tab:d_kap} sind die gemessenen Längen $l$ und Durchmesser $d$ der Kapillare.
Aus den Werten in Tabelle \ref{tab:mohrsche_waage} und der Formel \eqref{eq:mohrsche_waage} folgen die relativen Dichten von destilliertem Wasser, Methanol und Ethylenglykol und mit der Voraussetzung $\rho _{Wasser} = 10^{3}\cdot\kilogramm\cdot\meter^{-3}$ die Dichten in Tabelle \ref{tab:dichte}.

\begin{table}[!h]
\centering
\begin{tabular}{r|l}
    Kapillar & $l[\milli\meter]$\\
    \hline
    Rot & $231(1)$\\
    Blau & $220(1)$\\
    Grün & $228(1)$\\
    
 \end{tabular} 
 \caption{\label{tab:l_kap}Länge $l$ der Kapillare}
\end{table}

\begin{table}[!h]
\centering
\begin{tabular}{r|c|c|c||c|c}
    Kapillar & Messung 1 & Messung 2 & Messung 3 & Verwendet & rel. Fehler\\
    \hline
    Rot & $69(2)$ & $84(2)$ & $86(2)$ & $85(2)$ & $ 2.4\%$\\
    Blau & $116(2)$ & $117(2)$ & $117(2)$ & $117(2)$ & $ 1.7\%$\\
    Grün & $181(2)$ & $179(2)$ & $181(2)$ & $180(2)$ & $ 1.1\%$\\
    
 \end{tabular} 
 \caption{\label{tab:d_kap} Durchmesser der Kapillare in $ 10^{-5} \meter$}
\end{table}

\begin{table}[!h]
\centering
\begin{tabular}{r|c|c|c|c|c|c|c|c|c}
    Auslenkung $r[\centi\meter]$ & 1 & 2 & 3 & 4 & 5 & 6 & 7 & 8 & 9\\
    \hline
    \hline
    Wasser & & & & & & & & $11$ & $100$ \\
    \hline
    Methanol & & & $1$ & $10$ & & & & $100$ & \\
    \hline
    Ethylenglykol & & & & $100$ & & $100$ & & $10$ & \\
    
 \end{tabular} 
 \caption{\label{tab:mohrsche_waage}Gewichteverteilung bei der Mohrschen Waage}
\end{table}

\begin{table}[!h]
\centering
\begin{tabular}{r|c}
    Flüssigkeit & Dichte $\rho [\kilogramm\cdot\meter^{-3}]$\\
    \hline
    Wasser & $1.00\cdot 10^{3}$\\
    \hline
    Methanol & $ 0.85\cdot 10^{3}$\\
    \hline
    Ethylenglykol & $ 1.09 \cdot 10^{3}$\\
    
 \end{tabular} 
 \caption{\label{tab:dichte}Dichte}
\end{table}

\subsection{Kapillarität}

Die Messwerte der Höhenunterschiede $h_{Kap}$ zur Bestimmung der Kapillarität sind in Tabelle \ref{tab:h_kap_was} für destilliertes Wasser, in Tabelle \ref{tab:h_kap_met} für Methanol und in Tabelle \ref{tab:h_kap_eth} für Ethylenglykol aufgeführt.
Aus diesen Messwerten ergeben sich nach der Formel \eqref{eq:oberflaechenspannung} die Werte für die Oberflächenspannung $\sigma$  in der Tabelle \ref{tab:oberflaechenspannung}, sowie in der Grafik \ref{img:ofs}.
Für die Berechnung der Fehler in der Oberflächenspannung gilt
\begin{align}
  \frac{\sigma_\sigma}{\sigma} = \sqrt{\left(\frac{\sigma_{R}}{R}\right)^2 + \left(\frac{\sigma_{\rho}}{\rho}\right) ^2 + \left(\frac{\sigma_{h}}{h}\right)^2} 
\end{align}

\begin{table}[!h]
\centering
\begin{tabular}{r|c|c|c}
    Kapillar & Messung 1 & Messung 2 & Messung 3\\
    \hline
    Rot & $35$ & $37$ & $38$ \\
    Blau & $9$ & $23$ & $24$ \\
    Grün & $15$ & $17$ & $15$ \\
    
 \end{tabular} 
 \caption{\label{tab:h_kap_was}$h_{Kap} \rm{in} \milli\meter$ für destilliertes Wasser mit einem Fehler von $\pm 1\milli\meter$}
\end{table}

\begin{table}[!h]
\centering
\begin{tabular}{r|c|c|c}
    Kapillar & Messung 1 & Messung 2 & Messung 3\\
    \hline
    Rot & $15$ & $14$ & $14$\\
    Blau & $10$ & $10$ & $11$ \\
    Grün & $6$ & $6$ & $7$\\
    
 \end{tabular} 
 \caption{\label{tab:h_kap_met}$h_{Kap} \rm{in} \milli\meter$ für Methanol mit einem Fehler von $\pm 1\milli\meter$}
\end{table}

\begin{table}[!h]
\centering
\begin{tabular}{r|c|c|c}
    Kapillar & Messung 1 & Messung 2 & Messung 3\\
    \hline
    Rot & $22$ & $22$ & $24$\\
    Blau & $20$ & $16$ & $15$ \\
    Grün & $10$ & $10$ & $11$\\
    
 \end{tabular} 
 \caption{\label{tab:h_kap_eth}$h_{Kap} \rm{in} \milli\meter$ für Ethylenglykol mit einem Fehler von $\pm 1\milli\meter$}
\end{table}

\begin{table}[!h]
\centering
\begin{tabular}{r|c|c|c}
    Kapillar/Messung & $\sigma _{Wasser}$ & $\sigma _{Methanol}$ & $\sigma _{Ethylenglykol}$\\
    \hline
    Rot 1 & $ 73(3)$ & $ 27(2)$ & $ 50(3)$\\
    Rot 2 & $ 77(3)$ & $ 25(2)$ & $ 50(3)$\\
    Rot 3 & $ 79(3)$ & $ 25(2)$ & $ 55(3)$ \\
    \hline
    Blau 1 & $ 26(3)$ & $ 24(3)$ & $ 63(3)$ \\
    Blau 2 & $ 66(3)$ & $ 24(3)$ & $ 50(3)$ \\
    Blau 3 & $ 69(3)$ & $ 27(3)$ & $ 47(3)$ \\
    \hline
    Grün 1 & $ 66(5)$ & $ 23(4)$ & $ 48(5)$ \\
    Grün 2 & $ 75(5)$ & $ 23(4)$ & $ 48(5)$ \\
    Grün 3 & $ 66(5)$ & $ 26(4)$ & $ 53(5)$ \\
    
 \end{tabular} 
 \caption{\label{tab:oberflaechenspannung}Oberflächenspannung $\sigma$ in $\milli\newton\cdot\meter^{-1}$}
\end{table}

\begin{figure}[!h]
\centering
\includegraphics[scale=0.6]{ofs.png}
\caption{\label{img:ofs} Gemessene Oberflächenspannungen}
\end{figure}

Damit ergibt sich 
\begin{table}[!h]
\centering
\begin{tabular}{r|c|c|c}
    & Wasser & Methanol & Ethylenglykol\\
    \hline
    Gewichteter Mittelwert & $66.1\pm1.1$ & $25.1\pm0.8$ & $51.8\pm1.1$ \\
    \hline
    Literaturwert & $72.8$ & $22.6$ & $31.4$ \\
 \end{tabular} 
 \caption{Gewichtete Mittelwerte}
\end{table}

\subsection{Viskosität}

Aus der Formel \eqref{eq:viskositaet}
\begin{align}
 \dot V&=\frac{R^{4}\cdot\pi\cdot(p_1-p_2)}{8l\cdot\eta} \nonumber \\
 \folgt \frac{\Delta V}{\Delta t} &= \frac{R^{4}\cdot\pi\cdot\Delta p}{8\cdot l\cdot\eta} \nonumber \\
 \folgt \eta &= \frac{R^{4}\cdot\pi\cdot\Delta p\cdot t_A}{8\cdot l\cdot\Delta V}
\end{align}

und den gemessenen Ausflusszeiten $t_A$ in Tabelle \ref{tab:t_A} von $\Delta V=0.205 \liter$ Wasser und einem Druckunterschied $\Delta p = h\cdot\rho\cdot g = (4,66\pm0,24)\cdot 10^{3} \pascal$ ergeben sich die Viskositäten in Tabelle \ref{tab:eta}.
Für die Fehler ergibt sich nach Fehlerfortpflanzung
\begin{align}
  \frac{\sigma_\eta}{\eta} = \sqrt{\left(4\cdot\frac{\sigma_{R}}{R}\right)^2 + \left(\frac{\sigma_{\Delta p}}{\Delta p}\right)^2 + \left(\frac{\sigma_{t_A}}{t_A}\right) ^2 + \left(\frac{\sigma_{l}}{l}\right)^2}  \nonumber
\end{align}

\begin{table}[!h]
\centering
\begin{tabular}{r|l}
    Kapillar & $t_A [\second]$\\
    \hline
    Rot & 67(1)\\
    Blau & 20(1)\\
    Grün & 6(1)\\
    
 \end{tabular} 
 \caption{\label{tab:t_A}Ausflusszeit $t_A$ von $20.5\milli\liter$ destilliertem Wasser bei $50\centi\meter$ Füllhöhe}
\end{table}

\begin{table}[!h]
\centering
\begin{tabular}{r|l}
    Kapillar & $\eta [\milli\pascal\cdot\second]$\\
    \hline
    Rot & $0.84\pm0.09$\\
    Blau & $0.95\pm0.09$\\
    Grün & $1.5\pm0.3$\\
    
 \end{tabular} 
 \caption{\label{tab:eta}Viskosität $\eta$ nach den Messwerten in Tabelle \ref{tab:t_A}}
\end{table}

Daraus ergibt sich der gewichtete Mittelwert $\bar{\eta}=0.93(6) \milli\pascal\cdot\second$. \\
Der Literaturwert\footnote{Nach P. Schaaf (2014): ”Das Physikalische Praktikum”. Universitätsverlag Göttingen} für die Viskosität von Wasser zum Vergleich bei $20\degree\celsius$ liegt bei $\eta = 1.002 \milli\pascal\cdot\second$.
\\
\\
Um einen zweiten Wert für die Viskosität zu erhalten kann nach der Formel 
\begin{align}
 h(t) = h_0 \cdot \rm{exp}\left(-\frac{\rho\cdot g\cdot R^{4}}{8\cdot\eta\cdot l \cdot r^{2}}\cdot t \right)
\end{align}
mit den Werten aus Tabelle \ref{tab:t(h)} eine lineare Regression von $log\left(h(t)\right)$ über $t$ durchführen und bekommt $f(t)=m\cdot t+b$ mit
\begin{align}
 m=-\frac{\rho\cdot g\cdot R^{4}}{8\cdot\eta\cdot l\cdot r^{2}} \label{eq:m}
\end{align}
wobei $V=h\cdot\pi\cdot r^{2}$

\begin{table}[!h]
\centering
\begin{tabular}{r|c|c|c|c|c|c|c|c|c|c}
     Messung & $1$ & $2$ & $3$ & $4$ & $5$ & $6$ & $7$ & $8$ & $9$ & $10$\\
    \hline
    \hline
    $h[\centi\meter]$ & $45$ & $40$ & $35$ & $30$ & $25$ & $40$ & $30$ & $43$ & $42$ & $38.5$ \\
    \hline
    $t(h)[\second]$ & $52(1)$ & $127(2)$ & $210(3)$ & $309(4)$ & $421(5)$ & $142(1)$ & $324(2)$ & $52(1)$ & $108(2)$ & $162(3)$ \\
    
 \end{tabular} 
 \caption{\label{tab:t(h)}Abflusszeit $t(h)$}
\end{table}

\begin{figure}[!h]
\centering
\includegraphics[scale=0.5]{Viskositaet.png}
\caption{Höhe $h$ der Wassersäule halblogarithmisch über der Zeit $t(h)$}
\end{figure}

Als Geradensteigung ergibt sich $m=-(1.60\pm0.01) \second^{-1}$. Nach der Formel \eqref{eq:m} folgt $\eta = 0.83(8) \milli\pascal\cdot\second$ mit der Formel
\begin{align}
 \frac{\sigma_{\eta}}{\eta}= \sqrt{\left(4\cdot\frac{\sigma_{R}}{R}\right)^2 + \left(\frac{\sigma_{V}}{V}\right)^2 + \left(\frac{\sigma_{l}}{l}\right) ^2 + \left(\frac{\sigma_{m}}{m}\right)^2}  \nonumber
\end{align}
für den Fehler.
\section{Diskussion}
\label{sec:diskussion}
Im Allgemeinen ist festzustellen, dass die Kapillare teilweise nicht zentriert und nicht sonderlich rund gebohrt sind. Besonders die Exzentrizität zusammen mit der Tatsache, dass die Kapillare beim Messen des Durchmessers nicht fixiert sind, kann zu großen Fehlern führen. Ein Stoß gegen den Tisch hat warscheinlich unsere erste Messung des Durchmessers des roten Kapillar verfälscht. Deshalb haben wir diese Messung im Weiteren nicht berücksichtigt.
\\
Bei der Berechnung der Oberflächenspannungen sind relativ große Fehler gegenüber den Literaturwerten aufgetreten. Dies lässt sich durch Schwerigkeiten beim Ablesen von $h_{kap}$ aufgrund zu kleiner und verschmierter Skalen erklären. Außerdem sind wahrscheinlich trotz gründlicher Reinigung Verunreinigungen in den Kapilaren zurückgebileben, denn teilweise hat sich der Flüssigkeitsspiegel beim Bewegen des Kapillar nicht verändert. So ist es wahrscheinlich zu der starken Abweichung u.a. bei der vierten Messung von Wasser gekommen.  
\\
Die Ergebnisse für die Viskosität sind recht annehmbar, denn trotz vieler potentieller Fehlerquellen sind die relativen Fehler mit ca. $6.9\%$ bzw. $9.8\%$ verhältnismäßig klein und der Literaturwert bei der gegebenen Raumtemperatur von $(24.5\pm0.5) \degree\celsius$ liegt mit etwas über $0.89 \milli\pascal\cdot\second$ im Bereich der Messtoleranz.


\end{document}
